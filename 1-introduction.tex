\chapter{绪论}
%
\section{研究背景和意义}
%研究背景
近年来随着人工智能技术和社会经济的高速发展,大量信息化、智能化的新技术渗透到了人们的大众生活中。其中理解和预测人体运动相关研究获得了显著的进展。该技术被广泛应用与自动驾驶、智能机器人、人机交互和多媒体领域。在自动驾驶领域,车载计算机需要预测其他交通参与成员的行动意向和未来位置,并以此来规划车辆未来运行路线。在智能机器人领域,特别是用于协助人类的机器人,如工业机器人、看护机器人等,需要准确地预测人的未来运动来采取对应行动。在人机交互领域,在人口稠密的空间中,机器应准确预测周围的人的动作以安全地穿过人群。在多媒体领域,特别是游戏和影视制作场景中,和通过昂贵的动作捕捉设备获得人体运动模型相比,基于软件的理解和预测人体运动方法更加廉价高效。综上,理解和预测人体运动算法在促进国民经济发展和数字化、智能化转型方面有较高的研究价值。

%现阶段的研究和问题
目前学术界和工业界对该课题进行了较为细致且充分的研究。人体运动预测问题被定义为:在某个三维场景下,已知某个个体的一段历史运动序列,需要根据该段历史运动中包含的趋势或规律,预测该个体在未来的运动序列。该问题的研究重点包含两部分,第一是通过对历史运动序列的理解,提取其中包含的运动信息。例如在观看任意一段运动序列后,人类可以轻易地判别出该序列的运动类型(如行走、拾取物品、舞蹈等)。但对计算机来说,如何理解运动序列中的时序信息是研究的重点。第二是基于对历史运动序列信息的提取,预测未来运动序列。由于人体运动的高度复杂性和不确定性,如何基于有限的运动信息尽可能降低预测过程的不确定性,从而输出准确的未来运动序列,是当前研究的一个主要难点。

%当前的研究进展
在早期的研究中,由于循环神经网络(RNN\cite{zaremba2014recurrent})可以利用其内部隐状态(Hidden State)来捕捉输入数据的时间依赖性,对于处理时间序列这类连续数据特别有效,所以RNN被用来提取人体运动序列中的时序信息,预测未来的运动序列。

这类方法的主要思想是,每个RNN Unit有一个隐状态,可以在每个时间步骤中根据当前的输入的人体运动姿态和以前的隐状态进行更新。这个隐状态是对网络过去所见信息的总结,并被用来对未来进行预测。通过使用以前的隐状态来计算当前的隐状态,RNN可以捕获输入序列的时间依赖性。这使得RNN可以提取关于序列结构和序列元素之间的依赖关系的信息,这对于时间序列预测任务特别有利。

然而,由于RNN中每一步输出只与当前输入和上一步隐状态有关,无法对每一步输出进行整体约束。这导致输入序列和预测序列的过渡部分出现不连续的情况。为此现有方法提出了一种有着编码器-解码器结构的序列到序列模型(Sequence-to-Sequence),编码器将输入数据整体映射到隐空间,随后由解码器一次性预测未来运动序列,由此可以对输出进行全局一致性约束。此外,隐状态容量有限,RNN只能对短期依赖性进行建模,无法处理长距离时序依赖。这导致网络无法完全提取输入序列中的时序信息。为了解决这个问题,出现了长短期记忆(LSTM\cite{shi2015convolutional})和门控循环单元(GRU\cite{cho2014learning})网络,但它们更加复杂和计算量更大。
最后,该类方法通常将一个人体姿态作为一个整体输入RNN Unit,忽略了人体姿态的空间结构。然后,对于人体运动序列预测问题,人体姿态的空间结构是一个重要的先验信息。这导致基于RNN的方法在预测结果真实性和准确性方面有所欠缺。

近年来随着对图卷积网络(GCN\cite{kipf2016semi})的深入研究,部分现有方法引入图卷积网络对人体姿态空间结构进行建模。对于人体姿态这类不规则图状数据,图卷积网络有着天然的优势。在这类方法中,人体姿态被视作由一组顶点(或节点)和连接一对顶点的一组边组成的数学结构。通过图卷积网络对复杂的关节点对之间的联系进行建模。但传统图卷积网络主要应用于空间维度,如何设计高效的,具有时空信息提取能力的图卷积网络,对于人体运动序列预测这类涉及到时空序列数据的问题尤为重要,至今也依旧是学术界的一个难点问题。

除了需要尽可能提取输入序列中的时空信息,如何降低预测过程中的不确定性也是需要考虑的问题。在大多数情况下,由于人体运动序列的复杂性,输入序列和未来序列之间存在较大的差异,这导致预测过程存在较大的不确定性和预测歧义。现有方法大多采用单个前馈神经网络,直接接受输入序列,并预测未来运动序列。网络在预测过程中将承受较大的不确定性,预测结果可能出现模式坍塌(Mode collapse)问题。因此,如何设计更高效的预测策略来降低预测过程中的不确定性和歧义,是当前研究中需要重视的问题。

%我们的贡献
\section{主要研究内容及贡献}
针对上述研究存在的问题和人体运动姿态预测问题的特点,本文主要的研究内容和贡献被总结如下:

1.基于渐进式策略的人体运动姿态预测算法框架。
\begin{enumerate}[topsep = 0 pt, itemsep= 0 pt, parsep=0pt, partopsep=0pt, leftmargin=44pt, itemindent=0pt, labelsep=6pt]
	\item[$\bullet$] 与现有方法使用单阶段的网络结构不同,我们由浅至深地将预测过程拆分为多个阶段,除开位于网络入口的阶段,其他阶段均在上一步预测基础上进行预测,这将有利于降低每一阶段的预测难度。
	\item[$\bullet$] 我们遵循网络由浅至深,预测难度由易到难的原则。浅层阶段只负责预测大致的运动趋势,复杂的运动细节预测则由具有深层语义提取能力的深层阶段负责。
	\item[$\bullet$] 为了构建多阶段、渐进式的网络,我们为每个阶段构造对应的中级监督目标(Intermediate target)。具体的,我们设计了一种人体运动轨迹平滑方法,通过平滑关节点运动轨迹的方式,逐步去除运动细节,为每个阶段由深至浅提供不同平滑程度的预测目标。
\end{enumerate}

2.具有时空信息提取能力的Spatial-temproal 图卷积网络模块($SD−GCN$ 和 $TD−GCN$)。
\begin{enumerate}[topsep = 0 pt, itemsep= 0 pt, parsep=0pt, partopsep=0pt, leftmargin=44pt, itemindent=0pt, labelsep=6pt]
	\item[$\bullet$] 我们提出了一种新颖的具有时空信息提取能力的图卷积模块,该模块由时间信息提取模块和空间信息提取模块两个独立的图卷积构成。
	\item[$\bullet$] 时间信息提取模块称为$TD-GCN$,输入数据被视为多个关节点轨迹,$TD-GCN$提取时间维度上的信息。空间信息提取模块称为$SD-GCN$,输入数据被视为多个人体姿态,$SD-GCN$提取空间维度上 的信息。两个模式以串联的方式构成一个时空图卷积模块,当数据依次通过二者时,网络间接地提取到了时空信息。
	\item[$\bullet$] 由于时间和空间信息提取模块相互独立,因此随着输入数据的时间长度和空间复杂度提高,模型空间复杂度只随线性增长而非倍数增长,在保证模型信息提取能力的同时,降低了时间效率。
\end{enumerate}

3. 我们在三个公开数据集上使用通用度量指标,与现有先进方法进行对比。在预测精确性方面大幅领先(Human3.6M 6\%-7\% , CMU-MoCap 5\%-10\%, and 3DPW 13\%-16\%)。并且在时间效率和内存占用指标上我们也处于靠前位置。
