\chapter{绪论}
%
\section{研究背景和意义}
\label{section:1.1}
%研究背景
近年来随着人工智能技术和社会经济的高速发展,大量信息化、智能化的新技术渗透到了人们的日常生活中。其中理解和预测人体运动相关研究获得了显著的进展。该技术被广泛应用于自动驾驶、智能机器人、人机交互和多媒体领域。在自动驾驶领域,车载计算机需要预测其他交通参与成员的行动意向和未来位置,并以此来规划车辆未来运行路线。在智能机器人领域,特别是用于协助人类的机器人,如工业机器人、看护机器人等,需要准确地预测人的未来运动来采取对应行动。在人机交互领域,在人口稠密的空间中,机器应准确预测周围的人的动作以安全地穿过人群。在多媒体领域,特别是游戏和影视制作场景中,和通过昂贵的动作捕捉设备获得人体运动姿态模型相比。基于软件的,理解和预测人体运动方法更加廉价高效。综上,理解和预测人体运动算法在促进国民经济发展和数字化、智能化转型方面有重要的推动作用,意味着该课题有较高的研究潜力和应用价值。

%现阶段的研究和问题
目前学术界和工业界对该课题进行了较为细致且充分的研究。3D人体运动预测问题被定义为:在某个三维场景下,已知某个个体的一段历史运动序列,算法需要根据该段历史运动中包含的趋势或规律,预测该个体在未来的运动序列。该问题的研究重点包含两部分,第一是通过对历史运动序列的理解,提取其中包含的运动信息。例如在观看任意一段运动序列后,人类可以轻易地判别出该序列的运动类型(如行走、拾取物品、舞蹈等)。但对计算机来说,如何理解运动序列中的时序信息是研究的重点。第二是基于对历史运动中时空信息的提取,预测未来运动序列。由于人体运动的高度复杂性和不确定性,如何基于有限的运动信息尽可能降低预测过程的不确定性,从而输出准确的未来运动序列,是当前研究的一个主要难点。

%当前的研究进展
在早期的研究中,由于循环神经网络\parencite{zaremba2014recurrent}可以利用其内部隐状态(Hidden state)来捕捉输入数据中的时序依赖性,有利于处理时间序列这类序列化数据,所以RNN被用来提取人体运动序列中的时序信息,进而预测未来的运动序列。

这类方法的主要思路是,每个RNN Unit中的隐状态,可以根据当前的输入的人体运动姿态和上一阶段的隐状态进行更新。该隐状态是对网络过去所见信息的总结,并被用来对未来进行预测。借助节点中的隐变量的记忆和更新能力,RNN可以感知输入序列化信息中的时序依赖性。这使得RNN可以提取人体运动序列中的时序信息,并高效应用于未来序列的预测任务。

然而,由于RNN中每一步输出只与当前输入和上一步隐状态有关,无法对连续的输出结果进行整体上的一致性约束。输入序列和预测序列的过渡部分会出现不连续的情况。为此,现有方法提出了一种有着编码器-解码器结构的序列到序列的网络模型(Sequence-to-Sequence),编码器将输入数据整体映射到隐空间,随后由解码器一次性预测未来运动序列。以此便可以对输出进行全局一致性约束。除了过渡部分不连续的问题,由于隐状态容量有限,RNN只能对短期依赖性进行建模,无法处理长距离时序依赖。为了解决这个问题,出现了长短期记忆\parencite{shi2015convolutional}(Long Short-Term Memory,LSTM)和门控循环单元\parencite{cho2014learning}(Gate Recurrent Unit,GRU)网络,但它们更加复杂,带来了更多的运行开销。
同时,该类方法通常将人体姿态作为一个整体输入RNN Unit,忽略了人体姿态的空间结构。然后,对于人体运动序列预测问题,人体姿态的空间结构是一个重要的先验信息。这导致基于RNN的方法在预测结果真实性和准确性方面有所欠缺。

近年来随着对图卷积网络\parencite{kipf2016semi}的深入研究,部分现有方法引入图卷积网络对人体姿态空间结构进行建模。对于人体姿态这类不规则图状数据,图卷积网络有着天然的优势。在这类方法中,人体姿态被视作由节点集合和连接顶点的边集合组成的数据结构。图卷积网络借助拉普拉斯矩阵描述关节点之间的信息流动状态,进而对复杂的节点对间的关联进行建模。但传统图卷积网络主要应用于二维平面数据结构。如何设计高效的,具有时空信息提取能力的图卷积网络,对于人体运动序列预测这类涉及时空序列数据的问题尤为重要,至今也依旧是学术界的一个难点问题。

除了需要有效提取输入序列中的时空信息,如何降低预测过程中的不确定性也是需要考虑的问题。在大多数情况下,由于人体运动序列的复杂性,输入序列和未来序列之间存在较大的差异,这导致预测过程存在较大的不确定性和预测歧义。现有方法大多采用单个前馈神经网络,接受历史运动序列,随后直接预测未来运动序列。在这一过程中,网络将承受较大的不确定性,预测结果可能出现模式坍塌(Mode collapse)和预测失准等问题。因此,设计更高效的预测策略来降低预测过程中的不确定性和歧义,是当前研究中需要重视的另一个问题。

%我们的贡献
\section{主要研究内容及贡献}
针对上述研究存在的问题和人体运动姿态预测问题的特点,本文主要的研究内容和贡献被总结如下:

1.基于渐进式策略的人体运动姿态预测算法框架。
\begin{enumerate}[topsep = 0 pt, itemsep= 0 pt, parsep=0pt, partopsep=0pt, leftmargin=44pt, itemindent=0pt, labelsep=6pt]
	\item[$\bullet$] 与现有方法使用单阶段的网络结构不同,我们将预测过程拆分为多个阶段,除开位于网络入口的阶段,其他阶段均在上一步预测基础上进行预测,这将有利于降低每一阶段的预测难度。
	\item[$\bullet$] 我们遵循网络由浅至深,预测难度由易到难的原则。浅层阶段只负责预测大致的运动趋势,复杂的运动细节预测则由具有深层语义提取能力的深层阶段负责。
	\item[$\bullet$] 为了构建多阶段、渐进式的网络,我们为每个阶段构造对应的中级监督目标(Intermediate target)。具体的,我们设计了一种人体运动轨迹平滑方法,通过平滑关节点运动轨迹的方式,去除运动细节,为每个阶段提供不同平滑程度的预测目标。
\end{enumerate}

2.具有时空信息提取能力的Spatiotemproal 图卷积网络模块(S−DGCN 和 T−DGCN)。
\begin{enumerate}[topsep = 0 pt, itemsep= 0 pt, parsep=0pt, partopsep=0pt, leftmargin=44pt, itemindent=0pt, labelsep=6pt]
	\item[$\bullet$] 我们提出了一种新颖的具有时空信息提取能力的图卷积模块,该模块由时间信息提取模块和空间信息提取模块这两个独立的图卷积构成。
	\item[$\bullet$] 时间信息提取模块称为T-DGCN,输入数据被视为关节点轨迹集合,T-DGCN提取时间维度上的信息。空间信息提取模块称为S-DGCN,输入数据被视为人体姿态集合,S-DGCN提取空间维度上的信息。两个模式以串联的方式构成一个时空图卷积模块,当数据依次通过二者时,网络间接地提取到了时空信息。
	\item[$\bullet$] 由于时间和空间信息提取模块相互独立,因此随着输入数据的时间长度和空间复杂度提高,模型空间复杂度线性增长而非倍数增长,在保证模型信息提取能力的同时,提高了时间效率。
\end{enumerate}

3. 我们在三个公开数据集上使用通用度量指标,与现有先进方法进行对比。在预测精确性方面大幅领先(Human3.6M 6\%-7\% , CMU-MoCap 5\%-10\%和 3DPW 13\%-16\%)。并且在时间效率和内存占用指标上我们也处于靠前位置。

\section{论文结构}
本文包括七个主要章节,包含绪论、相关工作、图卷积网络基础、基于渐进式策略的多阶段人体运动姿态预测框架、基于时空分离策略的Non-Local时空图卷积模块、实验和总结与展望部分。其中各个部分的主要内容安排如下:

首先,在第一章,对3D人体运动估计问题的研究背景和研究意义进行详细阐述。其次概述本文的主要研究内容和贡献。

第二章详细介绍了3D人体运动估计问题的研究现状和发展历程,对当前研究的参考价值。

第三章主要介绍图卷积网络相关基础知识,包含物理意义解读和图卷积公式推导。为详细叙述基于时空分离策略的Non-Local时空图卷积模块打下基础。

第四章提出了基于渐进式策略的多阶段网络框架,本文将从实验和直观分析的角度来叙述该设计的合理性和有效性,并介绍网络的详细结构。

第五章提出了基于时空分离策略的Non-Local时空图卷积模块,在这里我们通过回顾现有图卷积模块的得失,阐述ST-DGCN的设计逻辑和具体实现细节。

第六章中,我们在三个公开数据集(Human3.6M、CMU-MoCap、3DPW)上对基于渐进式策略的多阶段人体运动姿态预测算法进行定性和定量的对比,证明本方法大幅领先现有SOTA方法。后续,我们将在消融实验中对模型的各个模块进行定量分析,验证本方法的设计合理性。

最后一章是对本文进行总结与展望,对本文提出的基于渐进式策略的多阶段人体运动姿态预测算法进行概略性的总结。同时分析本方法的不足,对后续研究提出展望。



