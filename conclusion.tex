\chapter{总结与展望}
人体运动姿态序列预测算法近十年在学术界和工业界都得到了广泛的研究,该算法可以被应用于国民经济信息化、智能化建设的方方面面,例如自动驾驶、智能机器人、人机交互和多媒体领域。而当前在这该算法上的研究受限于预测过程中的不确定性和特征提取算子时空信息提取能力的缺失,在预测精度和执行效率方面任然存在不足。本文在充分调研和分析现有方法后提出了一种名为“基于渐进式策略的人体运动姿态预测算法”来解决上述问题。

具体的,本方法有两点创新,第一,针对复杂长时人体运动姿态预测过程中不确定性大的问题,我们提出使用渐进式策略,设计一个多阶段网络,将整体的预测任务分割为多个子任务,每个阶段只需要在上一个阶段的基础上进行预测。整个预测过程遵循由难到易的原则,浅层网络只需要预测运动的大致趋势,而具体末端运动细节则由后续深层网络在此基础上完善。在多阶段网络模型中,中级监督目标的设计尤为重要。因为人体运动姿态数据属于不规则图数据,难以像图像一样可以简单地进行下采样。因此,我们提出了名为累积均值平滑的中级监督目标构造方法。在该方法中,我们保留了重要的空间结构先验信息,通过平滑关节点轨迹的方式降低运动的复杂程度。相比较基于高斯卷积核的平滑方法,该方法能避免过渡部分的不连续,提供更加具有层次化的中级监督目标,辅助不同阶段间的预测平滑过渡。

第二,正对现有特征提取算子时空信息捕捉能力弱的缺点,我们提出了一种新的图卷积模块称为ST-DGCN,该模块由两个串联的GCN构成,分别被称为S-DGCN和T-DGCN,前者负责处理空间上人体结构信息,后者负责处理时间维度上的关节点运动信息。推理时,特征首先通过S-DGCN被提取空间信息,随后被送入T-DGCN提取时间信息,网络由此间接地感知到了人体运动姿态序列中的时空特征。与现有特征提取算子相比,ST-DGCN通过时空分离和参数共享的形式降低了网络参数量,由于时间和空间维度均使用图卷积进行建模,网络在时空都具有全局感受野,拥有对时空长时依赖的建模能力。

随后我们通过详细的定性和定量实验分析,证明了本方法在预测精确度和时空效率两方面都领先现有先进方法。随后的烧蚀分析证明了我们的多阶段网络相比较现有的单阶段预测方法有较大的优势。ST-DGCN图卷积模块也优于现有的特征提取算子。综上,本文提出了基于渐进式策略的人体运动姿态预测算法在预测精确度和时空效率两个指标上全面领先现有方法,显著推动了该领域相关问题的研究进展。

但本方法也存在一些未解决的问题,可以作为未来的拓展研究方向。首先,在人体运动姿态序列预测问题中,预测的上限由输入数据中的信息决定,如果输入运动序列和待预测的运动序列之间的差距过大,或是待预测的运动序列过长,即使将输入序列中的有效信息完全利用也无法有效降低预测过程中的不确定性。在这种情况下片面追求预测的精确性已经成为了一个不可能完成的任务。近来一些生成式的方法为该问题提供了解决思路,它们通过在预测过程中添加一定的随机性,可以在接受同一个输入序列的情况下,生成多个不同的未来运动序列。该类方法不再片面追求预测的精确性,而是兼顾预测结果真实性。

当前研究的另一个不足,是缺乏一个对结果客观评估的度量指标,现有指标MPJPE计算预测关节点与真实关节点间的平均误差,作为度量预测质量的标准。然而该方法对所有关节点一视同仁,无法反映关节点之间的差异(一般来说,我们更关注运动模式复杂的末端关节点,而不是运动平稳躯干关节点)。目前有部分方法采用了主观评价或基于判别器的方法,为解决这个问题提供了新思路。