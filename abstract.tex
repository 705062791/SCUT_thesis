\chapter{摘\texorpdfstring{\quad}{}要}
	3D人体运动姿态预测(3D Human motion prediction)指:在3D空间中,根据历史人体运动姿态序列,预测未来的人体运动姿态序列。随着人工智能化浪潮的到来,该技术被广泛应用于自动驾驶、监控视频异常检测、人体动作捕捉生成等领域中,有着良好的应用前景和研究价值。例如在自动驾驶算法中,需要根据行人当前运动轨迹来预测其未来运动趋势,进而指导自动驾驶程序做出相应处置。
	
	本文提出了一种新颖的3D人体运动姿态预测算法,与现有方法相比,本方法在预测精确度和运行效率的综合指标上有大幅领先。目前现有方法大多使用单个网络直接预测未来运动姿态。由于输入人体运动姿态序列与未来人体运动姿态序列之间普遍存在较大的差异。使用单阶段网络直接预测时,往往出现模式坍塌、预测失准等情况。本文提出了一种新颖的渐进式多阶段网络,将直接预测拆分为多阶段预测,允许神经网络逐步学习复杂的人体结构特征和关节点运动模式。具体的,本文设计了一种被称为累积均值平滑(Accumulate average smooth,AAS)的中级监督目标构造方法,通过平滑关节点运动轨迹,在保留人体空间结构信息的同时,降低了关节点运动的复杂度。凭借AAS,可以在由浅至深的各个网络阶段,构造由易到难、平滑过渡的预测目标。允许网络逐步完善预测结果。另外,网络中的特征提取模块也对预测精确性有较大影响。现有方法大多使用卷积神经网络(Convolutional Neural Network,CNN)、循环神经网络(Recurrent neural network,RNN)、图卷积网络(Graph Convolutional Network,GCN)。在人体运动姿态问题中,输入数据同时包含具有空间拓扑结构的人体姿态和时间序列上的关节点轨迹。图卷积网络以其对空间拓扑数据的优秀建模能力得到了广泛的关注。但目前仍然缺乏一种同时对时空维度进行高效建模的特征提取方法。为此,我们提出了一种具有时空信息捕捉能力的图卷积,被称为ST-DGCN,该图卷积由空间和时间两部分构成,分别称为S-DGCN(Spatial \ Dense \ Graph \ Convolution)和T-DGCN(Temporal \ Dense \ Graph \ Convolution), 两部分串行组合,当运动序列输入后,首先由S-DGCN提取空间信息,随后送入T-DGCN提取时间信息,由此网络间接地捕捉了时空信息,并具有全局感受野。
		
	在渐进式网络结构和S-DGCN、T-DGCN的帮助下,本方法在Human3.6M、CMU-MoCap、3DPW这三个公开数据集上使用公开度量指标,预测精度较现有方法均有较大提升,且运行效率和空间占用无显著落后。


\keywordsCN{3D人体运动姿态预测、渐进式策略、时空序列、图卷积网络}

\chapter{Abstract}
3D human motion prediction refers to predicting the future human motion sequence in 3D space based on the historical human motion posture sequence. With the arrival of the AI wave, this technology has been widely used in fields such as autonomous driving, abnormal detection in surveillance videos, and human motion capture generation, with good application prospects and research value. For example, in autonomous driving algorithms, it is necessary to predict the future movement trend of pedestrians based on their current motion trajectory, and then guide the autonomous driving program to make corresponding arrangements.

This article proposes a novel 3D human motion prediction algorithm, which outperforms existing methods in terms of prediction accuracy and operational efficiency. Currently, most existing methods use a single network to directly predict future motion postures. However, due to the significant difference between the input human motion posture sequence and the future human motion posture sequence, pattern collapse and prediction errors often occur when using a single-stage network for direct prediction. This article proposes a novel progressive multi-stage network that splits direct prediction into multiple stages, allowing the neural network to gradually learn complex human structure features and joint motion patterns. Specifically, this article designs an intermediate supervision objective construction method called Accumulate Average Smooth (AAS), which smoothes joint motion trajectories while retaining human spatial structural information and reducing joint motion complexity. With AAS, smooth transition prediction targets from easy to difficult can be constructed at each network stage, allowing the network to gradually improve the prediction results. Additionally, the feature extraction module in the network also has a significant impact on prediction accuracy. Most existing methods use Convolutional Neural Networks (CNN), Recurrent Neural Networks (RNN), and Graph Convolutional Networks (GCN). In the problem of human motion posture, the input data contains both human posture with spatial topological structure and joint trajectory on the time sequence. GCN has received widespread attention for its excellent modeling capability of spatial topological data. However, there is still a lack of feature extraction methods that efficiently model both temporal and spatial dimensions. Therefore, we propose a graph convolution with the ability to capture spatiotemporal information called ST-DGCN, which consists of two parts: S-DGCN (Spatial Dense Graph Convolution) and T-DGCN (Temporal Dense Graph Convolution). These two parts are combined in series. When the motion sequence is input, the spatial information is first extracted by S-DGCN, and then the temporal information is extracted by T-DGCN, indirectly capturing spatiotemporal information and having a global receptive field.

With the help of the progressive structure and the two strategies of S-DGCN and T-DGCN, this method improves the prediction accuracy compared to existing methods on three public datasets: Human3.6M, CMU-MoCap, and 3DPW, using public metrics, while maintaining operational efficiency.

\keywordsEN{3D human motion prediction、Progressive learning、Spatiotemporal sequence、Graph Convolutional Networks}