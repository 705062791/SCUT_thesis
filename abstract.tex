\chapter{摘\texorpdfstring{\quad}{}要}
	3D人体运动估计(3D Human motion prediction)指:在3D空间中,根据历史人体运动姿态序列,预测未来的人体姿态运动序列。随着人工智能化浪潮的到来,该技术被广泛应用于自动驾驶、监控视频异常检测、人体动作捕捉生成等领域中,有着良好的应用前景和研究价值。例如在自动驾驶算法中,需要根据行人当前运动轨迹来预测其未来运动趋势,进而指导自动驾驶程序做出相应处置。
	
	本文提出了一种新颖且高效的3D人体运动估计算法,与现有方法相比,本方法在预测精确度和运行效率的综合指标上有较大的领先。具体地,我们在分析总结现有方法优劣的基础上提出了两个改进策略:(1)3D空间中的人体运动存在高度的复杂性和不确定性,给预测过程带来了很大的难度。现有方法往往采用单个网络直接预测。在运动模式简单,周期性的样本上能达到较高的预测精度。但在处理较为复杂且无明显规律的动作类型时,往往出现模式坍塌、预测失准等情况。为此我们提出了一种渐进式的网络结构来降低整体的预测难度:网络由多个阶段构成,每个阶段不再直接预测最终结果,而是在上一阶段的基础上完善预测结果。浅层的阶段负责预测运动的大概趋势,深层阶段则在此基础上完善预测细节,使预测结果向真值逐渐靠拢,同时减少各个阶段的预测难度。(2)3D人体运动数据同时具有时间和空间两维度,且人体姿态为无向不规则图结构,其内涵的空间先验结构信息极为重要。所以,网络中,特征提取模块的时空信息提取能力与网络性能密切相关,现有方法大多使用$CNN$(卷积神经网络)、$RNN$(循环神经网络)、传统$GCN$(图卷积网络),该类方法只适用于处理2维空间数据,难以捕捉时序联系。为此,我们提出了一种具有时空信息捕捉能力的GCN模块,该模块由空间和时间两部分构成,分别称为$S-DGCN(Spatial \ Dense \ Graph \ Convolution)$和$T-DGCN(Temporal \ Dense \ Graph \ Convolution)$, 两部分串行组合,当运动序列输入后,首先由$S-DGCN$提取空间信息,随后送入$T-DGCN$提取时间信息,由此网络间接地捕捉了时空信息,并具有全局感受野。
	
	在渐进式结构和$S-DGCN$、$T-DGCN$这两点改进措施的帮助下,本方法在$Human3.6M$、$CMU-MoCap$、$3DPW$这三个公开数据集上使用公开度量指标,预测精度较现有方法均有较大提升,且运行效率无显著落后。


\keywordsCN{3D人体运动估计、渐进式策略、时空序列、图卷积网络}

\chapter{Abstract}
3D human motion prediction refers to predicting future human motion sequences in 3D space based on historical human motion posture sequences. With the advent of the AI wave, this technology has been widely applied in fields such as autonomous driving, abnormal detection in monitoring videos, and generation of human motion capture. It has good application prospects and research value. For example, in the autonomous driving algorithm, it is necessary to predict the future motion trend of pedestrians based on their current motion trajectory, and then guide the autonomous driving program to make corresponding responses.

This article proposes a novel and efficient algorithm for 3D human motion prediction. Compared with existing methods, this method has a significant advantage in the comprehensive indicators of prediction accuracy and operational efficiency. Specifically, based on analyzing and summarizing the advantages and disadvantages of existing methods, we propose two improvement strategies: (1) Human motion in 3D space is highly complex and uncertain, which brings great difficulty to the prediction process. Existing methods often use a single network to predict directly, which can achieve higher prediction accuracy on simple and periodic samples of motion patterns. However, in dealing with more complex and irregular motion types, pattern collapse and prediction errors often occur. Therefore, we propose a progressive network structure to reduce the overall prediction difficulty: the network consists of multiple stages, each stage no longer directly predicts the final result, but perfects the predicted result based on the previous stage. The shallow stage is responsible for predicting the general trend of motion, while the deep stage perfects the prediction details, gradually approaching the true value of the prediction results and reducing the prediction difficulty of each stage. (2) 3D human motion data has both temporal and spatial dimensions, and human posture is an undirected and irregular graph structure. The spatial prior structural information implied in it is extremely important. Therefore, in the network, the spatio-temporal information extraction ability of the feature extraction module is closely related to the network performance. Existing methods mostly use CNN, RNN, and traditional GCN, which are only suitable for processing 2D spatial data and are difficult to capture temporal connections. Therefore, we propose a GCN module with spatio-temporal information capture ability, which consists of two parts: S-DGCN (Spatial Dense Graph Convolution) and T-DGCN (Temporal Dense Graph Convolution), which are combined in series. When the motion sequence is input, the spatial information is first extracted by S-DGCN and then the temporal information is extracted by T-DGCN. Therefore, the network indirectly captures spatio-temporal information and has a global receptive field.

With the help of the progressive structure and the two improvement measures of S-DGCN and T-DGCN, this method has significantly improved prediction accuracy compared with existing methods on three public datasets, Human3.6M, CMU-MoCap, and 3DPW, using public evaluation metrics, while the operational efficiency is not significantly lower.

\keywordsEN{3D Human Motion Prediction、Progressive Learning、Spatial Temporal Sequence、Graph Convolutional Networks}