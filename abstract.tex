\chapter{摘\texorpdfstring{\quad}{}要}
	3D人体运动估计(3D Human motion prediction)指:在3D空间中,根据历史人体运动姿态序列,预测未来的人体姿态运动序列。随着人工智能化浪潮的到来,该技术被广泛应用于自动驾驶、监控视频异常检测、人体动作捕捉生成等领域中,有着良好的应用前景和研究价值。例如在自动驾驶算法中,需要根据行人当前运动轨迹来预测其未来运动趋势,进而指导自动驾驶程序做出相应处置。
	
	本文提出了一种新颖且高效的3D人体运动估计算法,与现有方法相比,本方法在预测精确度和运行效率的综合指标上有较大的领先。具体地,我们在分析总结现有方法优劣的基础上提出了两个改进策略:(1)3D空间中的人体运动存在高度的复杂性和不确定性,给预测过程带来了很大的难度。现有方法往往采用单个网络直接预测。在运动模式简单,周期性的样本上能达到较高的预测精度。但在处理较为复杂且无明显规律的动作类型时,往往出现模式坍塌、预测失准等情况。为此我们提出了一种渐进式的网络结构来降低整体的预测难度:网络由多个阶段构成,每个阶段不再直接预测最终结果,而是在上一阶段的基础上完善预测结果。浅层的阶段负责预测运动的大概趋势,深层阶段则在此基础上完善预测细节,使预测结果向真值逐渐靠拢,同时减少各个阶段的预测难度。(2)3D人体运动数据同时具有时间和空间两维度,且人体姿态为无向不规则图结构,其内涵的空间先验结构信息极为重要。所以,网络中,特征提取模块的时空信息提取能力与网络性能密切相关,现有方法大多使用$CNN$(卷积神经网络)、$RNN$(循环神经网络)、传统$GCN$(图卷积网络),该类方法只适用于处理2维空间数据,难以捕捉时序联系。为此,我们提出了一种具有时空信息捕捉能力的GCN模块,该模块由空间和时间两部分构成,分别称为$SD-GCN(Spatial \ Dense \ Graph \ Convolution)$和$TD-GCN(Temporal \ Dense \ Graph \ Convolution)$, 两部分串行组合,当运动序列输入后,首先由$SD-GCN$提取空间信息,随后送入$TD-GCN$提取时间信息,由此网络间接地捕捉了时空信息,并具有全局感受野。
	
	在渐进式结构和$SD-GCN$、$TD-GCN$这两点改进措施的帮助下,本方法在$Human3.6M$、$CMU-MoCap$、$3DPW$这三个公开数据集上使用公开度量指标,预测精度较现有方法均有较大提升,且运行效率无显著落后。


\keywordsCN{3D人体运动估计、渐进式策略、时空序列、图卷积网络}

\chapter{Abstract}
	

\keywordsEN{3D Human Motion Prediction、Progressive Learning、Spatial Temporal Sequence、Graph Convolutional Networks}