\documentclass[unicode,master]{scutthesis} % 草稿封面,硕士则添加选项master,博士则去掉。使用正式封面时注释该行
%\documentclass[unicode,master,pdfcover]{scutthesis} %   % 论文正式封面,pdfcover为可选项,终稿再添加,使用草稿封面时注释该行
\usepackage{fontspec,color,array,longtable,graphicx} 
\usepackage{anyfontsize} %消除字体警告
\usepackage{enumitem}
%%%%%%%%%%%%%%%%%%%%%%%%%%%%%%%%%%%%%%%%%%%%%%%%%%%%%%%%%%%%%%%%%%%%%%%%%%%%%%%%%%——by MCH
%编译范围
% \includeonly{chapter04}
%参考文献设置
\usepackage[backend=biber,style=gb7714-2015,gbalign=gb7714-2015,gbpub=false,gbnamefmt = lowercase]{biblatex}
\addbibresource[location=local]{MyLibrary.bib} % 如果在其他盘,改为相对路径。比如F盘,改为:F/MyLibrary.bib
\addbibresource[location=local]{mybibfile2.bib} % 无论什么来源的bib文件,只要由参考文献的BibTeX组成,都可以使用此模板。参考文献的BibTeX获取方法可百度
%页眉页脚设置
\usepackage{fancyhdr}
\usepackage{listings}
\usepackage{xunicode}
\renewcommand{\lstlistingname}{列表}
\pagestyle{fancy}
\fancyfoot[C]{\headfont\thepage}
\renewcommand{\chaptermark}[1]{\markboth{\chaptername\ #1}{}}
\renewcommand{\sectionmark}[1]{\markright{\thesection\ #1}}
\fancyhead[RE]{}
\fancyhead[RO]{}
\fancyhead[LE]{}
\fancyhead[LO]{}
\fancyhead[CO]{\headfont{\leftmark}}
\fancyhead[CE]{\headfont{华南理工大学硕士学位论文}}% 
\renewcommand{\headrulewidth}{1.5pt}
\renewcommand{\footrulewidth}{0pt}
%%%%%%%%%%%%%%%%%%%%%%%%%%%%%%%%%%%%%%%%%%%%%%%%%%%%%%%%%%%%%%%%%%%%%%%%%%%%%%%%%%
\usepackage[unicode=true,bookmarks=true,bookmarksnumbered=true,bookmarksopen=false,breaklinks=false,pdfborder={0 0 1},backref=false,colorlinks=true]{hyperref}
\hypersetup{pdftitle={LaTeX模板使用说明},
	pdfauthor={蒙超恒},
	pdfsubject={华南理工大学硕士学位论文},
%%	pdfsubject={华南理工大学博士学位论文},
	pdfkeywords={PDF关键字1;PDF关键字2},
%%		linkcolor=black, anchorcolor=black, citecolor=black, filecolor=black, menucolor=black, urlcolor=black, pdfstartview=FitH}% 黑白,提交版
	linkcolor=blue, anchorcolor=black, citecolor=red, filecolor=magenta, menucolor=red, urlcolor=magenta, pdfstartview=FitH}% 彩色

\makeatletter
%%%%%%%%%%%%%%%%%%%%%%%%%%%%%% LyX specific LaTeX commands.
\providecommand{\LyX}{\texorpdfstring%
	{L\kern-.1667em\lower.25em\hbox{Y}\kern-.125emX\@}
	{LyX}}
%% Because html converters don't know tabularnewline
\providecommand{\tabularnewline}{\\}
\makeatother
\begin{document}
	%%%%%%%%%%%%%草稿封面设置%%%%%%%%%%%%%使用“正式封面”时不需要理会这部分
	\title{基于渐进式策略的人体运动姿态预测算法}	
	\author{马铁铮}	
	\supervisor{指导教师:聂勇伟\ 副教授}	
	\institute{华南理工大学}	
	\date{2023年5月20日}
	%%%%%%%%%%%%%%%%%%%%%%%%%%%%%%%%%%%%%
	\maketitle	
	\frontmatter	%此后为罗马数字页码,页面类型为plain
	\chapter{摘\texorpdfstring{\quad}{}要}
	3D人体运动姿态预测(3D Human motion prediction)指:在3D空间中,根据历史人体运动姿态序列,预测未来的人体运动姿态序列。随着人工智能化浪潮的到来,该技术被广泛应用于自动驾驶、监控视频异常检测、人体动作捕捉生成等领域中,有着良好的应用前景和研究价值。例如在自动驾驶算法中,需要根据行人当前运动轨迹来预测其未来运动趋势,进而指导自动驾驶程序做出相应处置。
	
	本文提出了一种新颖的3D人体运动姿态预测算法,与现有方法相比,本方法在预测精确度和运行效率的综合指标上有大幅领先。目前现有方法大多使用单个网络直接预测未来运动姿态。由于输入人体运动姿态序列与未来人体运动姿态序列之间普遍存在较大的差异。使用单阶段网络直接预测时,往往出现模式坍塌、预测失准等情况。本文提出了一种新颖的渐进式多阶段网络,将直接预测拆分为多阶段预测,允许神经网络逐步学习复杂的人体结构特征和关节点运动模式。具体的,本文设计了一种被称为累积均值平滑(Accumulate average smooth,AAS)的中级监督目标构造方法,通过平滑关节点运动轨迹,在保留人体空间结构信息的同时,降低了关节点运动的复杂度。凭借AAS,可以在由浅至深的各个网络阶段,构造由易到难、平滑过渡的预测目标。允许网络逐步完善预测结果。另外,网络中的特征提取模块也对预测精确性有较大影响。现有方法大多使用卷积神经网络(Convolutional Neural Network,CNN)、循环神经网络(Recurrent neural network,RNN)、图卷积网络(Graph Convolutional Network,GCN)。在人体运动姿态问题中,输入数据同时包含具有空间拓扑结构的人体姿态和时间序列上的关节点轨迹。图卷积网络以其对空间拓扑数据的优秀建模能力得到了广泛的关注。但目前仍然缺乏一种同时对时空维度进行高效建模的特征提取方法。为此,我们提出了一种具有时空信息捕捉能力的图卷积,被称为ST-DGCN,该图卷积由空间和时间两部分构成,分别称为S-DGCN(Spatial \ Dense \ Graph \ Convolution)和T-DGCN(Temporal \ Dense \ Graph \ Convolution), 两部分串行组合,当运动序列输入后,首先由S-DGCN提取空间信息,随后送入T-DGCN提取时间信息,由此网络间接地捕捉了时空信息,并具有全局感受野。
		
	在渐进式结构和S-DGCN、T-DGCN这两个策略的帮助下,本方法在Human3.6M、CMU-MoCap、3DPW这三个公开数据集上使用公开度量指标,预测精度较现有方法均有较大提升,且运行效率无显著落后。


\keywordsCN{3D人体运动姿态预测、渐进式策略、时空序列、图卷积网络}

\chapter{Abstract}
3D human motion prediction refers to predicting the future human motion sequence in 3D space based on the historical human motion posture sequence. With the arrival of the AI wave, this technology has been widely used in fields such as autonomous driving, abnormal detection in surveillance videos, and human motion capture generation, with good application prospects and research value. For example, in autonomous driving algorithms, it is necessary to predict the future movement trend of pedestrians based on their current motion trajectory, and then guide the autonomous driving program to make corresponding arrangements.

This article proposes a novel 3D human motion prediction algorithm, which outperforms existing methods in terms of prediction accuracy and operational efficiency. Currently, most existing methods use a single network to directly predict future motion postures. However, due to the significant difference between the input human motion posture sequence and the future human motion posture sequence, pattern collapse and prediction errors often occur when using a single-stage network for direct prediction. This article proposes a novel progressive multi-stage network that splits direct prediction into multiple stages, allowing the neural network to gradually learn complex human structure features and joint motion patterns. Specifically, this article designs an intermediate supervision objective construction method called Accumulate Average Smooth (AAS), which smoothes joint motion trajectories while retaining human spatial structural information and reducing joint motion complexity. With AAS, smooth transition prediction targets from easy to difficult can be constructed at each network stage, allowing the network to gradually improve the prediction results. Additionally, the feature extraction module in the network also has a significant impact on prediction accuracy. Most existing methods use Convolutional Neural Networks (CNN), Recurrent Neural Networks (RNN), and Graph Convolutional Networks (GCN). In the problem of human motion posture, the input data contains both human posture with spatial topological structure and joint trajectory on the time sequence. GCN has received widespread attention for its excellent modeling capability of spatial topological data. However, there is still a lack of feature extraction methods that efficiently model both temporal and spatial dimensions. Therefore, we propose a graph convolution with the ability to capture spatiotemporal information called ST-DGCN, which consists of two parts: S-DGCN (Spatial Dense Graph Convolution) and T-DGCN (Temporal Dense Graph Convolution). These two parts are combined in series. When the motion sequence is input, the spatial information is first extracted by S-DGCN, and then the temporal information is extracted by T-DGCN, indirectly capturing spatiotemporal information and having a global receptive field.

With the help of the progressive structure and the two strategies of S-DGCN and T-DGCN, this method improves the prediction accuracy compared to existing methods on three public datasets: Human3.6M, CMU-MoCap, and 3DPW, using public metrics, while maintaining operational efficiency.

\keywordsEN{3D human motion prediction、Progressive learning、Spatiotemporal sequence、Graph Convolutional Networks} % 中英文摘要
	%%%%%%%%%%%%%%%%%%%%%%%%%%%%%%%%%%%%%%%%%%%%%%%%
	% 目录、表格目录、插图目录这几个字本身的大纲级别是一级的,即和章名有相同的字号字体。目录表的内容通过titletoc宏包在。cls文件设置了。
	%\cleardoublepage % pdfbookmark可能需要这一条才能正常工作
	\pdfbookmark{\contentsname}{toc} %为目录添加pdf文件书签
	\tableofcontents	%目录
	% \listoffigures	%插图目录(可选)
	% \listoftables	%表格目录(可选)
	
	\begingroup
		\renewcommand*{\addvspace}[1]{}
		\newcommand{\loflabel}{图} 
		\renewcommand{\numberline}[1]{\loflabel~#1\hspace*{1em}}	
		\listoffigures
		
		\newcommand{\lotlabel}{表}
		\renewcommand{\numberline}[1]{\lotlabel~#1\hspace*{1em}}
		\listoftables
	\endgroup

	%%%%%%%%%%%%%%%%%%%%%%%%%%%%%%%%%%%%%%%%%%%%%%%%%
	\include{symbols}	% 符号对照表(可选)
	\include{abbreviation} 	% 缩略词	
	
	\mainmatter %此后为阿拉伯数字页码
	
    %%%%%%%%%%%%%%%%%%%%%%%%%%%%%%%%%%%%%%%%%%%%%%页眉页脚设置 ——by MCH 
    \fancypagestyle{plain}{
    	\pagestyle{fancy}
    }	% 每章的第一页会默认使用plain,没有页眉。通过重定义plain为fancy解决
    \pagestyle{fancy}	%设置页眉页脚为fancy
    %%%%%%%%%%%%%%%%%%%%%%%%%%%%%%%%%%%%%%%%%%%%%%分章节,结合导言区的\includeonly命令可仅编译部分章节,编译时不用切换界面,直接在相应章节编译即可。
	\include{chapter01}%第一章
	\include{chapter02}%第二章
	\include{chapter03}%第三章
	\include{chapter04}
	% 自行根据需要添加章节。

	\backmatter %章节不编号但页码继续
	%%%%%%%%%%%%%%%%%%%%%%%%%%%%%%%%%%%%%%%%%%%%%%%%%%%%%%%%%%%%%%    微调,使得后续章节的页眉不带章号——by MCH
	\renewcommand{\chaptermark}[1]{\markboth{#1}{}}
	%%%%%%%%%%%%%%%%%%%%%%%%%%%%%%%%%%%%%%%%%%%%%%%%%%%%%%%%%%%%%%
	\chapter{总结与展望}
人体运动姿态序列预测算法近十年在学术界和工业界都得到了广泛的研究,该算法可以被应用于国民经济信息化、智能化建设的方方面面,例如自动驾驶、智能机器人、人机交互和多媒体领域。而当前在这该算法上的研究受限于预测过程中的不确定性和特征提取算子时空信息提取能力的缺失,在预测精度和执行效率方面任然存在不足。本文在充分调研和分析现有方法后提出了一种名为“基于渐进式策略的人体运动姿态预测算法”来解决上述问题。

具体的,本方法有两点创新,第一,针对复杂长时人体运动姿态预测过程中不确定性大的问题。本文提出使用渐进式策略,设计一个多阶段网络,将整体的预测任务分割为多个子任务,每个阶段只需要在上一个阶段的基础上进行预测。整个预测过程遵循由难到易的原则,浅层网络只负责预测运动的大致趋势,而具体末端运动细节则由后续深层网络在此基础上完善。在多阶段网络模型中,中级监督目标的设计尤为重要。由于人体运动姿态数据属于不规则图数据,难以像图像一样可以简单地进行下采样。因此,本文提出了名为累积均值平滑的中级监督目标构造方法。在该方法中,本文保留了重要的空间结构先验信息,通过平滑关节点轨迹的方式降低运动的复杂程度。相比较基于高斯卷积核的平滑方法,该方法能避免过渡部分的不连续,提供更加具有层次化的中级监督目标,辅助不同阶段间的预测平滑过渡。

第二,针对现有特征提取算子时空信息捕捉能力弱的缺点,本文提出了一种新的图卷积模块称为ST-DGCN,该模块由两个串联的GCN构成,分别被称为S-DGCN和T-DGCN,前者负责处理空间上人体结构信息,后者负责处理时间维度上的关节点运动信息。训练时,特征首先通过S-DGCN被提取空间信息,随后被送入T-DGCN提取时间信息,网络由此间接地感知到了人体运动姿态序列中的时空特征。与现有特征提取算子相比,ST-DGCN通过时空分离和参数共享的形式降低了网络参数量,由于时间和空间维度均使用图卷积进行建模,网络在时空都具有全局感受野,拥有对时空长时依赖的建模能力。

随后本文通过详细的定性和定量实验分析,证明了本方法在预测精确度和时空效率两方面都领先现有先进方法。随后的消融实验证明了本文的多阶段网络相比较现有的单阶段预测方法有较大的优势。ST-DGCN图卷积模块也优于现有的特征提取算子。综上,本文提出了基于渐进式策略的人体运动姿态预测算法在预测精确度和时空效率两个指标上全面领先现有方法,显著推动了该领域相关问题的研究进展。

但本方法也存在一些未解决的问题,可以作为未来的拓展研究方向。首先,在人体运动姿态序列预测问题中,预测的上限由输入数据中的信息决定,如果输入运动序列和待预测的运动序列之间的差距过大,或是待预测的运动序列过长,即使将输入序列中的有效信息完全利用也无法有效降低预测过程中的不确定性。在这种情况下片面追求预测的精确性已经成为了一个不可能完成的任务。近来一些生成式的方法为该问题提供了解决思路,它们通过在预测过程中添加一定的随机性,可以在接受同一个输入序列的情况下,生成多个不同的未来运动序列。该类方法不再片面追求预测的精确性,而是兼顾预测结果真实性。

当前研究的另一个不足,是缺乏一个对结果客观评估的度量指标,现有指标MPJPE计算预测关节点与真实关节点间的平均误差,作为度量预测质量的标准。然而该方法对所有关节点一视同仁,无法反映关节点之间的差异(一般来说,本文更关注运动模式复杂的末端关节点,而不是运动平稳躯干关节点)。目前有部分方法采用了主观评价或基于判别器的方法,为解决这个问题提供了新思路。 %结论
	 %%%%%%%%%%%%%%%%%%%%%%%%%%%%%%%%%%%%%%%%%%%%%% bibtex参考文献设置  (原版)
%%	\bibliographystyle{scutthesis}
%%	\bibliography{F:/MyLibrary}
	%%%%%%%%%%%%%%%%%%%%%%%%%%%%%%%%%%%%%%%%%%%%%%
	%%%%%%%%%%%%%%%%%%%%%%%%%%%%%%%%%%%%%%%%%%%%%% biber参考文献设置	——by MCH
	%\renewcommand*{\bibfont}{\refbodyfont}			% 设置文献著录字号比正文小一号(五号),需要小四号请注释该行. % 不推荐使用small,而是使用cls文件中精确定义了的字号。
	\phantomsection % “目录”中的链接能正确跳转,需要添加 \phantomsection 否则点击参考文献会跳转到结论
	\addcontentsline{toc}{chapter}{参考文献}	%目录中添加参考文献
	\printbibliography	% 参考文献著录
 	%%%%%%%%%%%%%%%%%%%%%%%%%%%%%%%%%%%%%%%%%%%%%%
 	% 只有一个附录
% 	\include{appendix}
 	% 有多个附录
	\include{appendix1} %附录1
	\include{appendix2} %附录2
 	%%%%%%%%%%%%%%%%%%%
	\chapter{攻读博士/硕士学位期间取得的研究成果} %博士/硕士记得选其一
\pubfont % 论文撰写规范里,这章是5号宋体,\pubfont 设置字号为5号了。但其实很多论文用小四号也OK。
一、已发表(包括已接受待发表)的论文,以及已投稿、或已成文打算投稿、或拟成文投稿的论文情况\underline{\textbf{(只填写与学位论文内容相关的部分):}}
\begin{table}
	\centering{}%
	\pubfont 
	\begin{longtable}{|>{\centering}m{0.5cm}|m{1.8cm}|>{\centering}m{2.8cm}|>{\centering}m{2.5cm}|>{\centering}m{2.2cm}|>{\centering}m{2.cm}|>{\centering}m{1cm}|}
		\hline 
		\textbf{序号} & \textbf{作者(全体作者,按顺序排列)} & \textbf{题 目} 						   & \textbf{发表或投稿刊物名称、级别} & \textbf{发表的卷期、年月、页码} & \textbf{与学位论文哪一部分(章、节)相关} &\textbf{被索引收录情况}\tabularnewline
		\hline 
		1    & 马铁铮、聂勇伟、龙成江、张青、李桂清& Progressively generating better initial guesses towards next stages for high-quality human motion prediction & Proceedings of the IEEE/CVF Conference on Computer Vision and Pattern Recognition(CCF A类会议)& 已录用,2022年3月,页码6437--6446 & 第四章、第五章 & EI\tabularnewline
		\hline 
	\end{longtable}
\end{table}

注:在“发表的卷期、年月、页码”栏:

1.如果论文已发表,请填写发表的卷期、年月、页码;

2.如果论文已被接受,填写将要发表的卷期、年月;

3.以上都不是,请据实填写“已投稿”,“拟投稿”。

不够请另加页。

二、与学位内容相关的其它成果(包括专利、著作、获奖项目等)



%注:这部分一言难尽,我努力了很久都没有把这个表做好。感觉学校给的这个表的模板非常反人类。看国外大学的博士论文,那种像参考文献著录信息那样一行一行的,比较美观。而这个框框很难放文字进去。

\normalsize % \normalsize可以将下文调回和正文一样的字号,这个随个人喜好。注释掉的话,致谢就就跟随《攻读博士/硕士学位期间取得的研究成果》的字号。 %成果
	\chapter{致\texorpdfstring{\quad}{}谢}
%把下面文字替换
在即将完成研究生学业之际,我非常荣幸能够向各位老师和同学们表达我最深刻的感激之情。在这段时间里,您们给予了我无私的指导和关心,让我能够在学术和生活上都得到了莫大的帮助。特别是我的导师聂勇伟副教授,在我的研究方向、论文的思路和实现等方面给予了我深入的指导和帮助,使我得以更好地了解所研究领域的知识,提高了我的研究能力,开拓了我的学术视野。我将永远珍视您对我的帮助和支持,您的辛勤付出将成为我一生中不断前行的动力。

同时,我也要感谢华南理工大学计算机学院的各位老师和同学们。感谢您们在学术、生活和事业上的帮助和鼓励,感谢您们给予我的指导和支持。在学习和生活的过程中,我遇到了许多挑战和困难,但是有您们的陪伴和鼓励,我才能够克服这些困难,不断向前。

此外,我还要感谢我的家人和朋友们。他们一直是我人生道路上的坚实后盾,他们的支持和鼓励是我前进的动力。在我研究生生涯中,他们无私的付出和关爱让我倍感温暖和感激,我会时刻珍惜这份感情。

最后,我要感谢我的朋友们张军,胡益畅,陈海斌,刘知安。感谢他们在学习和生活中的支持和帮助,他们的陪伴让我的研究生生活更加充实而有意义。

再次感谢你们,是你们的支持和帮助让我走到了今天的成就。我会永远铭记你们的谆谆教诲和关心,不断努力学习和提高自己,在自己的岗位上为社会和人民做出更多的贡献。
%把上面文字替换

~\\

\begin{minipage}[t]{0.945\textwidth}%
	\begin{flushright}
		马铁铮\\
%		\today\\	% 自动时间
		2022年6月10日\\	%固定时间
		于华南理工大学
		\par\end{flushright}
\end{minipage}

 %致谢
\end{document}
